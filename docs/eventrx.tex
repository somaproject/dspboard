The EventRX Interface receives events from the DSP to place onto the
EventBus via the DSP's SPI interface operating in SPI MASTER mode.
Note that because we boot the DSP via SPI, we must mux access to the
actual SPI interface, which is done via the spimux component. 

We need our design to be small, so we're using distram to store the
inbound data. We thus perform minimal buffering, and rely on the DSP
itself to queue events. The DSP should always rate-limit its event
transmission to at most one per ecycle. On top of that, sometimes the
dspcontproc might wish to send an event, and so the DSP will
potentially be denied on a given ECYCLE. But once the DSP begins
transmitting an event, it takes roughly a full ECYCLE due to the slow
speed of the SPI interface. 

The timing would look something like this:

  ECYCLE                 ECYCLE             ECYCLE               ECYCLE
    |                      |                  |                    |
DSP  SEND EVENT ----------> 
FPGA                        Actually TX EVENT

So to enable full throughput the queue must be double-buffered.

We then use a FLAG to indicate FIFOFULL status. 

Interface: 

Data is transmitted via the SPI interface in 16-bit words, and thus
11 words will constitute a full event. So every 11 words we call the 
received data an event and GO.

We use 16 bit words so we can more easily use a 16-bit-wide register
file out of distram. 

